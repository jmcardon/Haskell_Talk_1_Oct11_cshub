\documentclass{beamer}
 
\usepackage[utf8]{inputenc}
\usepackage{amsmath,amsthm,amssymb}
\usepackage{graphicx}
\usepackage{listings}
\usepackage{minted}
\usepackage{color}
\usepackage[document]{ragged2e}
\usepackage[font=small,labelfont=bf]{caption}
\usepackage{bussproofs}
\usepackage{syntax}
\usepackage{tikz-cd}
\newcommand*\pipenotdiv{\,\nmid\,}
\newcommand*\pipediv{\,|\,}
\usetheme{Madrid}

\lstset{
  frame=none,
  xleftmargin=2pt,
  stepnumber=1,
  numbers=left,
  numbersep=5pt,
  numberstyle=\ttfamily\tiny\color[gray]{0.3},
  belowcaptionskip=\bigskipamount,
  captionpos=b,
  escapeinside={*'}{'*},
  language=haskell,
  tabsize=2,
  emphstyle={\bf},
  commentstyle=\it,
  stringstyle=\mdseries\rmfamily,
  showspaces=false,
  keywordstyle=\bfseries\rmfamily,
  columns=flexible,
  basicstyle=\small\sffamily,
  showstringspaces=false,
  morecomment=[l]\%,
}

 
%Information to be included in the title page:
\title{Fun with Lattices  }
\author{Jose Cardona}
\institute{OQuant}
\date{LambdaConf 2019}
\graphicspath{{/home/jm0x5c/Documents/presentations/LC/images/}}

\begin{document}
 
\frame{\titlepage}

\author[Jose, Cardona] % (optional, for multiple authors)
{}


\begin{frame}
\frametitle{Why Haskell?}

Haskell is:

\begin{itemize}
\item Pure! (Never worry about side effects again!)
\item Declarative (wholemeal programming).
\item Likely older than you! (And me!)
\item Statically typed, with type inference.
\item Functional! (unlike my attention span)
\end{itemize}

\end{frame}

\begin{frame}
\frametitle{Why not haskell}

For the most part, the reasons "not to use haskell" are similar to the reasons you'd avoid GC'ed languages in the first place:

\begin{itemize}
\item You're building a piece of software that has resource constraints (RTOS, AAA Games).
\item You're building a piece of software that has incredibly tight latency requirements (HFT).
\item You're not a fun person (but we can fix that, you're here! that's the first step)
\end{itemize}

\end{frame}

\begin{frame}[fragile]
\frametitle{A lil' haskell}

A simple declaration: We read this as "x has type Int"
\begin{center}
\begin{minted}{haskell}
x :: Int
x = 5
\end{minted}
\end{center}

Haskell comes equipped with a few basic types built-in:

\begin{itemize}
\item Int = Machine-dependent sized integer
\item Integer = Arbitrary precision integer  (Java folks will know it as bigInt)
\item Double, Float = Same as other languages (32/64 bit precision floats)
\item Lists (written as $[a]$ where a is some type)
\item n-ary tuples (a, b), (a, b, c), etc.
\end{itemize}


\end{frame}

\begin{frame}[fragile]
\frametitle{A lil' more haskell}

Let's try every programmer's favorite definition to start a language:  the factorial function.

\begin{minted}{haskell}
factorial :: Integer -> Integer
factorial 0 = 1
factorial n = n * factorial (n-1)
\end{minted}

You might notice a few peculiar things:
\begin{itemize}
\item There's "multiple lines" of the factorial function.
\item somehow there's a literal 0 on the left hand side.
\end{itemize}

The ability to "inspect the structure" of an argument is a particular concept called "pattern matching" which is central to the way you write haskell code. It's very similar to writing structurally inductive proofs.

\end{frame}

\begin{frame}[fragile]

Recall well formed formulae. An expression $A$ is a well formed formula if it is of the following form:

\begin{itemize}
\item it is an "Atomic" metavariable $p, q, r, s...$
\item It is of the form $(\neg A)$ where A is a WFF
\item It is of the form $A \circ B$ where A and B and WFF and $\circ$ is a logical connective 
\end{itemize}

Proving $ ((\neg p) \lor q) \land q$ is a WFF simply involves recursively traversing the statements (top down or bottom up) and building the WFF that way.

Pattern matching works on that same premise.

\end{frame}

\begin{frame}
\frametitle{Arithmetic Operators}
 	
\begin{itemize}
\item $+,-, *$ work as you expect, however: Haskell \textit{does not} have any sort of implicit numeric conversions. You may come to appreciate this later, but: to go from a fractional type to an integral type: use $round, ceiling, floor$. To go from an integral to a rational, use $fromIntegral$.
\item modulo is $mod$ (not $\%$).
\item $\/$ is for fractional division. For integral division, use $div$
\end{itemize}
\end{frame}

\begin{frame}[fragile]
\frametitle{Booleans}

\begin{itemize}
\item $True$ and $False$.
\item logical negation is just $not$. It's just a function.
\item $\&\&, ||$ logical operators
\item For ordering/equality: use $==, >, >=, <, <=$ though note: equality is structural equality in haskell, and comparison operators only operate on the same type.
\end{itemize}

\end{frame}

\begin{frame}[fragile]
\frametitle{Algebraic data types}

if you're familiar with enums in java:

\begin{minted}{java}
public enum Level {
 HIGH,
 LOW,
 EVEN_LOWER,
 INFO,
 SWEET_N_LO
}
\end{minted}

We have the same in haskell:

\begin{minted}{haskell}
data Level = High | Low | EvenLower | Info | SweetNLo
\end{minted}

Similar to enums: the compiler assists us quite a bit in determining if we're missing a case for our enums if we don't provide it. However,
 there's some power algebraic data types have that enums do not

\end{frame}


\begin{frame}[fragile]
\frametitle{Algebraic Data types}

\begin{itemize}
\item Algebraic data types can contain data.
\item Algebraic data types can be recursive.
\item Algebraic data types can be pattern matched on.
\end{itemize}

\begin{minted}{haskell}
data ArithExpr 
  = Add ArithExpr ArithExpr
  | Sub ArithExpr ArithExpr 
  | Value Int
\end{minted}

Example, with the above defined, we can now sum on this tree:

\begin{minted}{haskell}
eval :: ArithExpr -> Int
eval (Add l r) = eval l + eval r
eval (Sub l r) = eval l - eval r
eval (Value i) = i
\end{minted}

\end{frame}

\end{document}
